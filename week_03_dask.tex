\documentclass{article}
\usepackage[margin=0.75cm]{geometry}
\usepackage{listings}
\usepackage{float}
\usepackage{color}
\newcommand{\writing}[1]{\uwave{#1}}  
\newcommand{\tristan}[1]{\color{red}TG: #1}

 
\title{Dask: Parallel computation wit blocked algorithms and task scheduling}
\begin{document}
\maketitle

Dask provides a framework to describe a program as a direct acyclic graph (DAG) of tasks.
Each node of the graph is a task and each input edges is a dependency.
Therefore, each task needs to complete their dependency before being executed.
Then this DAG is given to a runtime that handles the whole execution.

The graph is represented by a python dictionary where keys are mapped to either values or tasks.
\textit{A task is a tuple with a callable first element}.
\textit{A key can be any hashable element that is not a task}\footnote{Those sentences are word for word what is writen in the paper, but it was so clear I had to use it. \tristan{It's fine but then use quotes or the quote Latex environment. And write something like
``as written by the authors: '\ldots'" }}.

Figure~\ref{fig:simple-python} shows a very simple algorithm and Figure~\ref{fig:simple-python-encode} shows how it can be stored in the dictionary \tristan{I understand why it's here, but an abstract shouldn't include code excerpts. 
Instead, it should summarize the paper with sentences.}.
\begin{figure}[H] 
	\begin{lstlisting}[language=python]
	def inc(i):
		return i + 1
	def add(a, b):
		return a + b
	x = 1
	y = inc(x)
	z = add(y, 10)
	\end{lstlisting}
	\caption{A simple python code}
	\label{fig:simple-python}
\end{figure} 
\begin{figure}[H] 
	\begin{lstlisting}[language=python]
	d = {'x': 1,
	'y': (inc, 'x'),
	'z': (add, 'y', 10)}
	\end{lstlisting}
	\caption{The dictionary of the previous simple python code.}
	\label{fig:simple-python-encode}
\end{figure} 

The main element to build this DAG is the direct use of \textit{Dask Arrays}.
Instead of creating a classic NumPy\footnote{NumPy is python package for scientific computing.} array, the user will create a \textit{Dask Arrays}.
Then, he will manipulate the array exactly like any NumPy array (adding values, summing the array, etc).
All the manipulations done to the arrays will add up to the DAG.
The last thing to do is to call the function \textit{compute} on \textit{Dask Array}.

Figure~\ref{fig:example-dask} shows an example of how to use \textit{Dask Arrays}.
We can see that a \textit{Dask Array} is used exactly like a usual array, but instead
of doing the actual computation, Dask will only build the DAG.
\begin{figure}[h] 
	\center
	\begin{lstlisting}[language=python]
	import dask.array as da
	x = da.arange(15)
	y = (x + 100)
	z = y.sum()
	z.compute()
	\end{lstlisting}
	\caption{An example of \textit{Dask arrays}. \textit{z} is initialized with \textit{x}, it is then stored as a
	\textit{Dask array}. Note that \textit{z} does not have a real value before the call to \textit{compute}, where all computation is scheduled and executed.}
	\label{fig:example-dask}
\end{figure} 

Dask also relies on blocked algorithms which are algorithms working with chunks of data rather than individual data.
Let us say, we would like to sum an array, we could add each element of the array to the next.
But, we could also split the array into four sub-array, compute the sum for each of them then sum the four sub-sum.
The second example is a blocked algorithm because it works with chunks of the original data.
Blocked algorithms help creating more parallelism because you can often execute the sub-computation in parallel.

Dask provides a parameter to split a \textit{Dask Array} into chunks when it is created.
Then, when creating the DAG, Dask will create one task for each chunk and one task to combine the results from them.

To execute the DAG, Dask uses dynamic schedulers.
Every time a task is completed, its state is updated and its output dependencies are updated as well.
Therefore, new tasks may become ready.

After several tries, it was found that a policy of \textit{last-in first-out} was more effective than others.
The reason exposed in the paper is because it releases intermediate results fast enough so there is no need to write them on the disk.

If no scheduler suits the need of the user, it is possible to implement a new scheduler.

Dask also provides two other parallel collections.
The first one is a \textit{bag}, which is a list without any guarantee on the order.
Dask bag follows the \textit{PyToolz} API.
The second collection is a dataframe that is close to \textit{Panda}.

To conclude, I would say that, like Spark, Dask put itself in between the user and the computation.
Thanks to that, it builds a graph of task dependency also known as DAG and this DAG is executed by a scheduler.
That makes Dask a tool easy to use even for scientific with a minimal background in programming.

\tristan{Your abstract correctly covers the paper and is at the right level of detail.
But it reads a bit like a bullet list: you should aim at getting a more compact, written flow.
This will help you when you write your literature review.}
\end{document}

