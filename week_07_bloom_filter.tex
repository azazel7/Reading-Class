\documentclass{article}
\usepackage[margin=0.75cm]{geometry}
\usepackage{listings}
\usepackage{float}


\title{Space/Time Trade-offs in Hash Coding with Allowable Errors}
\begin{document}
\maketitle
The membership problem is when an object or message must be identified as a member of a set previously created.
Previous methods used to store allowed messages and to test the received messages against the one already stored.
However, the set of messages stored may not fit in the main memory.
The paper discusses a new approach that requires less space to deal with this problem.
The gain in space is made by allowing a small fraction of error.

The paper talk about one conventional hashing methods and two hashing method with allowable errors.
The conventional method will first hash the message then store this message in an array
at the cell indexed by the hash.
Testing a new message for membership is like hashing the message and check if the message is stored
in the cell corresponding to its hash.
The first hashing method with allowable error (Method 1) is similar to the conventional method except
that instead of storing the message, it will store a fingerprint of this message. 
The adjustable error comes from the size of the fingerprint.
The smaller it is, the more error there will be.
The second method with allowable error (Method 2) works similarly to Method~1, but instead of storing
a fingerprint, it only stores one bit: 0 for an empty cell and 1 otherwise.
Thus the strength of the of Method~2 relies on the distribution of the hashing function.
To increase this strength, the paper proposes to use a hash function that returns multiple hash values.
Then those values are used to address many 1-bit cells in the array\footnote{We note that using one hash function
that return multiple values can also be seen as combining multiple hash function that return one value.}.
Those two methods with allowable errors should be used in a very specific case.
The conventional method should not fit in memory and a certain rate of false positive should be acceptable.
The false positive can be verified later by accessing a slower memory.
In order to make the best out of those methods, the number of messages identified as members of the set should be small
compared to the non-member messages. 
Further comparison in the paper between method~1 and method~2 demonstrates that the second method is more efficient.
The paper end on an experimental result of method~2 applied on hyphenation.

To conclude, the paper exposes a new approach to deal with the membership problem when the set of member messages
does not fit in memory. This new approach relies on allowing a small number of false positive to achieve a smaller
size for the set. This method is more efficient when a small fraction of the messages are members. Moreover, it does not prevent
to use a slower memory to store the whole set of messages.



\end{document}

%vim: tw=50 ts=2:

