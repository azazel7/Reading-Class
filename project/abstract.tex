\documentclass{article}
\usepackage[margin=0.75cm]{geometry}
\usepackage{listings}
\usepackage{graphicx}
\usepackage{hyperref}

 
\title{Big Data Project}
\author{Martin Khannouz}
\begin{document}
\maketitle

\section*{Abstract}
%a few sentences summarizing the document.
This document is a brief abstract about a Big Data project.
\section{Introduction}
The idea of an Internet of Things (IoT) has been growing during the past few years.
IoT can be seen as large objects (i.e watch, fridge or cars) connected to internet as
well as smaller sensors forming networks.
In the second case, it is wise to make the chip embeded as much energy efficient as
possible. It is known that the most energy consuming part of a communicating device
is the network. The less the device communicates, the longer its batery will last.

Hence, it may be interesting to leverage the computational capacity of a device
so it can mine its own data streams. Even though this capacity is small, a simple
chip can still execute basic operations like filtering, compression or sampling.
Then the device can decide which part of the data stream to send through the
network.

The objective of this project is to build a data stream library adapted to embeded
environment. This library will focus at providing simple data stream algorithm
for microcontroller with a FreeRTOS operating system.

\section{Related work}
Many libraries on data stream mining already exist. Some specialize on one
algorithm (i.e. the Bloom Filter, a reservoir sampling) and some others implement
a wider variety of data stream algorithm and data structure. In this section, we will have a brief
overview of existing projects that provide algorithms to mine data stream.

\paragraph{StreamingCC}
	\href{https://github.com/jiecchen/StreamingCC}{StreamingCC} is a C++ library for summarizing data streams.
	It provides several data stream algorithms such as Count-Min Sketch;
	Count-Sketch; Distinct Elements Counter; Reservoir sampling; Bloom Filter and its variants.
	However, StreamingCC makes a heavy use of standard library containers such as vectors.
	This use of the standard library makes it difficult to export with embedded systems like
	FreeRTOS because features provided by the kernel are not on the microcontroller.

\paragraph{Awesome Streaming}
	\href{https://github.com/manuzhang/awesome-streaming}{Awesome Streaming} is a GitHub repository that lists many streaming resources:
	framework; applications, or reading material.
	Not all resources target embedded software. Some are streaming engine relying on Spark and some other are frameworks for gathering
	data from sensor networks.

\paragraph{Bloom Filter}
	Many libraries implement the Bloom Filter data structure.
	This \href{http://www.partow.net/programming/bloomfilter/index.html}{library} is a
	well-documented C++ Bloom Filter Library that is implemented in a single header
	and does not rely on any dependencies.
	The \href{https://github.com/Baqend/Orestes-Bloomfilter}{Orestes-Bloomfilter}
	is a java library that implements four Bloom Filter variants.
	Two of them support concurrent access to the structure.
	\href{http://matthias.vallentin.net/blog/2011/06/a-garden-variety-of-bloom-filters/}{library of Bloom Filter (libbf)}
	provides a wide variety of Bloom Filter based algorithms. It includes the Spectral Bloom Filter; the Stable Bloom Filter; the $A^2$ Bloom Filter; and
	the bitwise Bloom Filter.
	The library \href{https://github.com/jvirkki/libbloom}{libbloom} provides an implementation in plain C for the basic Bloom Filter.
	Finally, \href{https://github.com/brianmadden/rust-bloom-filter}{rust-bloom-filter} is a Rust implementation of the Bloom Filter
	and it includes MurmurHash3.

\paragraph{Cuckoo Filter}
	The Cuckoo filter is a Bloom Filter alternative. Given the parameters and the environment,
	it may achieve better performance than the Bloom Filter.
	Here is a non-exhaustive list of GitHub repositories that implement a Cuckoo Filter many languages: \href{https://github.com/vijayee/cuckoo-filter}{javascript};
	\href{https://github.com/seiflotfy/rust-cuckoofilter}{Rust};
	\href{https://github.com/seiflotfy/cuckoofilter}{Go};
	and \href{https://github.com/begeekmyfriend/CuckooFilter}{C}. 

\paragraph{Flajolet-Martin}
	The Flajolet-Martin algorithm aims at counting distinct element in a stream.
	This \href{https://github.com/svengato/FlajoletMartin}{repository} provides
	an implementation of this algorithm that uses boost and
	FarmHash\footnote{\href{https://github.com/google/farmhash}{FarmHash} is an open-source library that implements hash functions.}

\paragraph{Reservoir Sampling}
	Reservoir Sampling is a group of algorithms that sample a stream.
	The standard algorithm replaces an element in the sample with the new element with a certain probability.
	Variants exist with weight elements. 
	\href{https://www.geeksforgeeks.org/reservoir-sampling/}{Here} is a brief code that applies a reservoir sampling algorithm to integers.
	The \href{https://en.cppreference.com/w/cpp/algorithm/sample}{standard C++ library} provides may use reservoir sampling in the sample algorithm.
	\href{http://erikerlandson.github.io/blog/2015/11/20/very-fast-reservoir-sampling/}{Here} is a scala code that implements a reservoir sampling method.

\section{Materials and Methods}
Given that the library is destined for microcontroller and sensors we will limit ourselve to datasets
with simple datatype: temperature; humidity; acceleration.
The library will be developped so it can compile and execute in a FreeRTOS project.
The algorithm that will be implemented will be two filtering algorithms and two sampling methods:

\begin{itemize}
	\item Bloom Filter
	\item Cuckoo Filter
	\item Basic Reservoir Sampling.
	\item Chained Reservoir Sampling.
\end{itemize}

\end{document}
