\documentclass{article}
\usepackage[margin=0.75cm]{geometry}
\usepackage{listings}
\usepackage{float}


\title{The PageRank Citation Ranking: Bringing Order to the Web}
\begin{document}
\maketitle
The paper presents the \textit{PageRank} method
and how it can be used to improve search engines
of the early World Wide Web. This method not
only relies on the text of each webpage but
rather on the metadata information about them.

The paper defines the forward link of a webpage $u$
as hypertext link on that wepage toward another webpage while the
backward links are the links from other pages that
point to the webpage $u$. The \textit{PageRank}
method propose to use the backward links as an
indicator of the importance of a page.
A webpage $u$ can compute its current rank based
on the ranks of the webpages that point to it,
hence its backlinks. In the same vein, a webpage
$u$ can be seen as distributing its ranks between all
its forward links.
To compute the webpage's ranks, the algorithm
will initialize their value then refine
it iteratively based on the backlink ranks of the webpage. 
The initial values of the ranks have little
impact on the result although a good
initialization may speedup the process.
The paper briefly discuss about issues implies by
this method like the huge amount of data to
compute, the proper way to ignore ranks of
webpage without forward link or how to deal with
closed loops in the graph.

After that description of the \textit{PageRank},
the paper explains how the search engine Google is
built upon this metric. Given a keywords, the
search engine will gather all webpage title
containing these keywords then sorts the result by
their \textit{PageRank} value.
Results presented in the paper show better
accuracy than other search engines.
Then the paper proposes a way of tweaking the
\textit{PageRank} equation to personalized the
query results. Such personalization can be made
base on the whereabouts of the user or by guessing
its interest.

To conclude, the paper describes a new method to
estimate the importance of webpages. This method
appears to be more efficient than previous ones
and also more robust to manipulation. The paper
describes a new search engine built upon this
metric and displays comparison with current search
engines. The results are promising and the paper
gives many insights on how this method could be
improved.

%the current graph of the crawlable Web has




\end{document}

% vim: tw=50 ts=2:


