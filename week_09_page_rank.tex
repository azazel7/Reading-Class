\documentclass{article}
\usepackage[margin=0.75cm]{geometry}
\usepackage{listings}
\usepackage{float}


\title{The PageRank Citation Ranking: Bringing Order to the Web}
\begin{document}
\maketitle
The paper present the \textit{PageRank} method in
order to improve search engine of the early World
Wide Web. This method not only relies on the text
of each webpage, but rather on the metadata
information about those pages.

The paper define the forward link of a webpage $u$
as hypertext link toward another webpage while the
backward links are the links from other pages that
point to the webpage $u$. The \textit{PageRank}
method propose to use the backward links as an
indicator of the importance of a page.
A webpage $u$ can compute its current rank based
on the ranks of the webpages that point to it,
hence its backlinks. In the same vein, a webpage
$u$ can be seen as distributing its ranks between all
its forward links.
To compute the webpage's ranks, the algorithm
will initialize their value then refine
it iteratively based on the backlink ranks of the webpage. 
The initial values of the ranks have little
impact on the result although a good
initialization may speedup the process.
The paper briefly discuss about issues implies by
this method like the huge amount of data to
compute, the proper way to ignore ranks of
webpage without forward link or how to deal with
closed loops in the graph.

After that description of the \textit{PageRank},
the paper explains how the search engine Google is
build upon this metric. Given a keywords, the
search engine will gather all webpage title
containing these keywords then sorts the result by
their \textit{PageRank} value.
Results presented in the paper show better
accuracy than other search engine.
Then the paper propose a way of tweaking the
\textit{PageRank} equation to personalized the
query results. Such personalization can be made
base on the whereabout of the user or by guessing
its interest.

%the current graph of the crawlable Web has




\end{document}

% vim: tw=50 ts=2:


