\documentclass{article}
\usepackage[margin=0.75cm]{geometry}
\usepackage{listings}
\usepackage{float}


\title{Mining Frequent Patterns without Candidate Generation}
\begin{document}
\maketitle
The paper propose a data scructure to address the Frequent Patterns problem.
This problem can be explained as follow.
Let $I = \{a_0, a_1, \cdots, a_m\}$ be a set of items and $DB = \{T_0, T_1, \cdots, T_n\}$ a transaction database where each $T_i$ is composed of items in $I$.
A frequent pattern is a sub-set of items $A_k = \{b_1, \cdots, b_p\}$ that occures at least $\lambda$ in $BD$ where $\lambda$ is user defined threshold.
Then mining frequent patterns is finding all frequent patterns in $DB$.
To solve this problem, the paper propose a compact data structure based on a tree that only need to read the database twice.

The tree is built in two steps.
The first steps read through all the transaction to count frequent items and keep a list of those that occurs more than $\lambda$.
The second step build the tree while reading the database a second time.
Each node in that tree contains an item and a counter.
The root is initialized with a null item.
Then, each transaction $T_i$ will be read from the database and trimmed of the less frequent items.
The remaining items will be sorted in a list $T_i'$ according to their frequency in the database.
This sorted list can be defined by $T_i' = [b_0, \cdots, b_p]$, where each $b_k \in T_i$ and the frequency of each $b_k$
is more than the threshold $\lambda$.
Then elements of $T_i'$ are inserted in the tree so each node containing the item $b_k$ is the parent of the node containing $b_{k+1}$.
The parent of $b_0$ is the root.
If a node containing $b_k$ already exists in the branch and at the level it should be inserted, then the counter of this node is incremented by one. 
At the end of the second step, all branches where nodes have a counter higher than $\lambda$ define a set of frequent patterns.
The whole set of frequent patterns associated to one branch are all combinations of nodes from that branch.




\end{document}

%vim: tw=50 ts=2:

