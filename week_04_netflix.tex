\documentclass{article}
\usepackage[margin=0.75cm]{geometry}
\usepackage{listings}
\usepackage{float}
 
\title{Matrix Factorization Techniques for Recommender Systems}
\begin{document}
\maketitle
The recommender systems presented in the paper is
based on a matrix factorization where rows and
column are products and users. Values in cells are
the rating gave by user $u$ for product $i$.
The idea is to find the user's preferences in
order to recommand a product suit for him.

Formally, each item $i$ (product or movie) is
associated with a vector $q_{i} \in R^{f}$.
$q_{i}$ is a vector of factors about the item
(author, type of movie, length, ...).
Each user is associated with a vector $p_{u} \in
R^{f}$. $p_{u}$ indicate the interest of the user
$u$ about each factor in $q_{i}$.
The between given by a user $u$ for a product $i$
is $r_{ui} = q_{i}^{T} p_{u}$. We call $r_{ui}$
the prediction for the the real rating $r_{ui}$.
A hard part in the recommender system is to gather
the value for $p_{u}$ and $q_{i}$ for all user and
all product because most of $p_{u}$ are not
available. Thus, the recommender system will learn
them. To do so, the algorithm will try to minimize
the regularized squared error.

There is two learning algorithms proposed in this
paper. The stochastic gradient descent will first
predict the rating $r_{ui}$ then compute the
error to adjust the values of both $p_{u}$ and $q_{i}$.
The second algorithm (Alternating Least Square) is
iterative and has the interesting feature of
being suit for massive parallelization.

To correctly recommand products, the recommender algorithm
has to face many issue. The biais issue is when a
product or a user is systematically over or under
rated. For example, when rating a movie, some user
give always good marks will other give always bad
marks. The same is true for a movie which will
receive better score because of an advertasing
campaigne. To counter those biais we introduce
$b_{ui} = \mu + b_{u} + b_{i}$. $\mu$ is the
overall average rating; $b_{u}$ and $b_{i}$ are
the observed deviation from the average. The
rating predicted become $r_{ui} = b_{ui} + q_{i}^{T} p_{u}$.
Another issue faced by the system is when there is
very few data about users. To compensate for that
lack of information, we can gather other
data about the user regardless of its
willingness~: zip code; age; gender; browsing
history; mouse movement.
User's preference ($p_{u}$ and $b_{u}$) and
product popularity ($b_{i}$) may change over
time. To adress this issue, $p_{u}$, $b_{u}$, and
$b_{i}$ are considered as function of the time
$t$ ($p_{u}(t)$, $b_{u}(t)$, and $b_{i}(t)$).
A last issue faced by the recommender system is
the influence on rating due to event outside the
range of the product. Those event could be a massive
advertasing about a movie or a user that
systematically over-rate one type of product
without viewing the content. To bypass this
problem, the system introduce a level of
confidence when a user rate a product. This level
of confidence is computed certain information like
\textit{how long did the user watch a show}
or \textit{how frequently does the user by a
product}. Then the level of confidence is use 
when computing the regularized square error to
weight the error given this level.
This type of recommender system has won the Netflix
prize competition in 2007 and 2008. 

\end{document}

%vim: tw=50 ts=2:
